\documentclass[12pt]{article}

\usepackage{amsmath}
\usepackage{amsfonts}
\usepackage{amssymb}
\usepackage{graphicx}
\usepackage{caption}
\usepackage{mathtools}
\usepackage{lipsum}
\usepackage{stackengine}
\usepackage{fancyhdr}
\usepackage{caption}
\usepackage{tikz}
\usetikzlibrary{shapes.geometric, arrows}
\usepackage{float}
\usepackage[a4paper,left=1in,right=1in,top=1in,bottom=1in,footskip=.25in]{geometry}
\usepackage{etoolbox}
\usepackage[nottoc]{tocbibind}
\usepackage{tabu}
\usepackage{enumitem,kantlipsum}
\usepackage{appendix}
\usepackage{pdfpages}
\begin{document}

	\begin{titlepage}
		\centering
		
		\begin{figure}
			\begin{center}
				\includegraphics[scale=.2]{tubaf.pdf}  
			\end{center}
			
		\end{figure}
		
		
		
		%\includegraphics[width=0.15\textwidth]{download}\par\vspace{1cm}
		{\scshape\LARGE \textbf{Technische Universit\"at Bergakademie Freiberg} \par}
		\vspace{1cm}
		{\scshape\Large PERSONAL PROGRAMMING PROJECT\par}
		{\scshape\Large (PPP)\par}
		\vspace{1.5cm}
		{\huge\bfseries Implementation of Gradient Elasticity Model In FEM \par}
		\vspace{2cm}
		{\scshape\Large \textbf{Dhaval Rasheshkumar Patel}\par}
		{\scshape\Large 63940\par}
		\vfill
		{\normalsize\ Supervised by\par}
		
		Dr.~ \textsc{Sergii Kozinov}
		
		\vfill
		
		% Bottom of the page
		{\large \today\par}
	\end{titlepage}
	\clearpage
    \textbf{\LARGE Abstract}\\ \newline
    \newline


    \newpage
    	\listoffigures
    \clearpage
    \tableofcontents
    \clearpage

\section{Introduction}
Classical continuum solid mechanics theories, such as linear or non-linear elasticity and plasticity, have been used in wide range of fundamental problems and applications in various fields, but "Classical Continuum Constitutive Models" possess no material/intrinsic scale. So in this regime of micron and nano-scales that experimental evidence and observations have suggested that classical continuum theories do not suffice for an accurate and detailed description in the modelling of size dependent phenomena. Moreover, classical elastic singularities as those emerging during the application of point loads and description of size effects also not solve by it.
\newline
\par
In General, in Gradient elasticity theories, the length scales enter the constitutive equations through the elastic strain energy function, which, in this case depends not only on the strain tensor but also on gradients of the strain tensors Gradient Elasticity theories provide extensions of the classical equations of elasticity with additional higher order spatial derivatives of strains and stresses, but in Gradient elasticity theory one of the most challenging task is to keep the number of additional constitutive parameters to a minimum.
\newline
\par
Stress equation of equilibrium, constitutive equations and boundary conditions of the "Strain Gradient theory" were first given in a non-linear form by Toupin in 1960s. After this famous Strain Gradient elasticity theory in three different forms proposed by Mindlin in 1960s. In such theories , when the problem is formulated in terms of displacements, the governing partial differential equation contains the fourth order derivative of displacements. If traditional finite element formulation are used for the numerical solution of such problems, then $C^{1}$ displacement continuity is required. $C^{1}$ displacement continuity means displacement and its first derivative is continuous in inter-element. However, in FEM there is no any robust $C^{1}$ continuous element. So an alternative mixed finite elements formulation is developed, in which both the displacement and the displacement gradients are used as independent unknowns and their relationship is enforced using Lagrange Multiplier Method. In 1999 JOHN Y. SHU proposed the mixed finite formulation based Toupin—Mindlin theories, in which only $C^{0}$ continuous element was used. After this in 2002 E. Amanatidou and N. Aravas also proposed the Mixed type FEM formulation for Toupin—Mindlin theories and also used $C^{0}$ continuous element.
\newline  
\par
In the present "Personal Programming Project" report, in the very first section we discussed the theoretical background of the strain gradient elasticity theory with its evolutions and different versions. Following to this section, we discussed in brief few of them theories. After this the Mixed type of FEM formulation with $C^{0}$ continuous element for Strain Gradient theory is discussed in details.
\newpage
 
\section{Theoretical Background}
In this Section the main focus is to understand the "Classical Strain Gradient Elasticity Theory" and motivation behind the evolution of Strain Gradient Elasticity theories. In this section the brief overview of Toupin's, Mindlin's, Eringen's and Aifantis theory is given. After this also discussed about different possible FEM formulation.  
     \begin{figure}[H]
     	\begin{center}
     	     	\includegraphics[scale=0.5]{file.png}
     	\end{center}
     	\caption{The history of Strain Gradient Theory}
     \end{figure}
\subsection{Mindlin's 1964 Theory.}
In the early 1960s, the Mindlin presented \cite{mindlin1968first} a theory of elasticity with Microstructure length, in which he distinguished between kinematic quantities on two scale micro and macro. As discussed in introduction that it is very challenging task to keep the number of constitutive parameters to a minimum. In this theory constitutive tensor contains 1764 coefficients in total, but for isotropic material its reduced drastically to amount 18. Mindlin present the Strain energy equation as,
\begin{equation}\label{first}
\begin{aligned}
U= &\frac{1}{2}\lambda\varepsilon_{ii}\varepsilon_{jj}+\mu\varepsilon_{ij}\varepsilon_{ij}+\frac{1}{2}b_1 \gamma_{ii} \gamma_{jj}+\frac{1}{2}b_2 \gamma_{ij} \gamma_{ij}+\frac{1}{2}b_3 \gamma_{ij} \gamma_{ji}+g_1\gamma_{ii}\varepsilon_{jj}
\\   
&+g_2(\gamma_{ij}+\gamma_{ji})\varepsilon_{jj}+a_1\kappa_{iik}\kappa_{kjj}+a_2\kappa_{iik}\kappa_{jkj}+\frac{1}{2}a_3\kappa_{iik}\kappa_{jjk}
\\
&+\frac{1}{2}a_4\kappa_{iij}\kappa_{ikk}
 +a_5\kappa_{iij}\kappa_{kik}
+\frac{1}{2}a_8\kappa_{iji}\kappa_{kjk}+\frac{1}{2}a_{10}\kappa_{ijk}\kappa_{ijk}
\\
&+a_{11}\kappa_{ijk}\kappa_{jki}+\frac{1}{2}a_{13}\kappa_{ijk}\kappa_{ikj}
+\frac{1}{2}a_{14}\kappa_{ijk}\kappa_{jik}
+\frac{1}{2}a_{15}\kappa_{ijk}\kappa_{kji}
\end{aligned}
\end{equation}  
\newline
where $\lambda$ and $\mu$ are the usual Lame constants and the various $a_i$ , $b_i$ and $g_i$ are 16 additional constitutive coefficients. However for practical purpose the use of eq(1) is very limited as it requires so much additional coefficients. In later 1960s Mindlin also formulated the simpler version of his own elasticity theory by making assumption of expressing the strain energy density in terms of displacement only. So in last he proposed three different form of it. However, the equation of strain energy density is of order four and it requires $C^{1}$ continuity.
\newpage

\subsection{Aifanti's 1992 Theory.} 
In the early 1990s, motivated by own work in plasticity and non-linear elasticity, Aifantis \cite{askes2011gradient} suggested to extend the linear constitutive model given by Mindlin in form $II$, in which second order terms are expressed as the strain gradient tensor. So he modified this Mindlin form $II$ by neglecting the most of the gradient coefficients. Mindlin present that for a general isotropic elastic solid the strain energy density depends upon
$\varepsilon_{ij}$ and $\kappa_{ijk}$ as the following,
\begin{equation}\label{two}
\begin{aligned}
W(\varepsilon,\kappa) = 
& G(\varepsilon_{ij}\varepsilon_{ij}+\frac{\nu}{1-2\nu}\varepsilon_{ij}\varepsilon_{ij}) \\
&    +a_1\kappa_{iik}\kappa_{kjj}+a_2\kappa_{iik}\kappa_{kjj}+a_3\kappa_{iik}\kappa_{kjj}+a_4\kappa_{iik}\kappa_{kjj}+a_5\kappa_{iik}\kappa_{kjj}
\end{aligned}
\end{equation}
where G is elastic shear modulus, $\nu$ is Poisson's ratio and $a_1$,$a_2$,$a_3$,$a_4$,$a_5$ are material constants.   
\newline
In this he considered,
\newline
\begin{equation}
\begin{aligned}
& a_1 = a_3 = a_5 = 0,   &   a_2 = \frac{\nu}{1-2\nu}Gl^2, &&  a_4 = Gl^2
\end{aligned}
\end{equation} 
Accordingly, the strain energy density is defined as
\begin{equation}
W = \frac{1}{2}\lambda\varepsilon_{ii}\varepsilon_{jj}+\mu\varepsilon_{ij}\varepsilon_{ij}+l^2(\frac{1}{2}\lambda\varepsilon_{ii,k}\varepsilon_{jj,k}+\mu\varepsilon_{ij,k}\varepsilon_{ij,k})
\end{equation}
In Eq. (4), $\lambda$ and $\mu$ are Lame constants, while $l$ represents microstructural parameter (material length scale).However this equation(4) of strain energy also requires 
$C^{1}$ continuous element.
\newline
\newline
\textbf{Conclusion of above both theories (2.1) and (2.2) : }
\newline
\newline
In above both strain gradient theories the principle of virtual work for a linear elastic strain gradient solid can be expressed as
\begin{equation}
\int\limits_\upsilon\! [\sigma_{ij} \delta\varepsilon_{ij} + \tau_{ijk}\delta\eta_{ijk}]  dV = \int\limits_\upsilon\! [b_k\delta u_k] dV + \int\limits_s \! [f_k\delta u_k+r_kD\delta u_k]dS
\end{equation}
where $D(.) = n_k\frac{\partial(.)}{\partial x_k}$ is surface normal-gradient operator, $b_k$ is the body force per unit volume
of the body $V$ while $f_k$ and $r_k$ are the truncation and the double stress traction per unit area of the surface $S$. They are in equilibrium with the Cauchy stress $\sigma_{ij}$ and the higher-order stress $\tau_{ijk}$ according to
\begin{equation}
b_k + (\sigma_{ij}-\tau_{jik,j})_{,i} = 0
\end{equation}
The constitutive law governing the stress $\sigma_{ij}$ and the higher-order stress $\tau_{ijk}$ for an elastic solid is derived through $\sigma_{ij} = \partial w / \partial \varepsilon_{ij}$ and $\tau_{ijk}= \partial w / \partial \eta_{ijk}$ where w is the strain energy density per unit volume. In this Cauchy stress $\sigma_{ij}$ is work conjugate to the strain $\varepsilon_{ij}$ and higher-order stress $\tau_{ijk}$ is work conjugate to the strain gradient  $\eta_{ijk}$.
\newline
\newline
Here, second-order derivatives of displacement occurred in the principle of virtual work Eq.(5), implying that displacement-based elements of $C^1$-continuity are indispensable in a finite element formulation. However there are no robust $C^1$ continuous elements were available at that time for the application of fem formulation of above mentioned both strain gradient theories.

\section{Different FEM Approach}
There are different FEM approach to the Strain Gradient Elasticity theories are available because of the fact that $C^1$ continuous elements are very difficult to formulate and on the other hand there are also a mixed finite element formulation of strain gradient elasticity derived, which only requires $C^0$ continuity.

\subsection{Tomislav Lesicar - Two scale FEM formulation} 
\subsubsection{ C1-Continuous Element}
The Aifantis strain gradient theory given in subsection (2.2) has been embedded into finite element framework by Tomislav Lesicar, Zdenko Tonkovic and Jurica Soric \cite{lesivcar2017two}. They used the three node triangular finite element named C1PE3 \cite{putar2017damage}. The element is shown in
Fig. It contains twelve degrees of freedom (DOF) per node, and it satisfies $C^1$ continuity with assumptions of the plane strains with unit thickness
\newline
     \begin{figure}[H]
     	\begin{center}
	      \includegraphics[scale=0.8]{Tri_Element.JPG}  
	    \end{center}  
        \caption{C1-Continuous Triangular Element}
     \end{figure}
\subsubsection{ Higher-order stress and strain gradient tensor} 
They used the same weak form of the finite element formulation as given by Eq.(5). Furthermore, using the basic finite element relations strain and stress tensors can be expressed as
\begin{equation}
\varepsilon = Bu,      \sigma=C\varepsilon
\end{equation}
\newline
where B is a matrix containing linear combinations of the first derivatives of the components of the shape function matrix, ̄C is an isotropic elastic constitutive matrix and u  is a displacement vector. Relating to Eq.(7) the strain gradient tensor is represented as

\[
\varepsilon_{,1}=
\begin{bmatrix}
\varepsilon_{11,1} \\
\varepsilon_{22,1} \\
2\varepsilon_{12,1} 
\end{bmatrix}
=B_{xx}u , \quad   
\varepsilon_{,2}=
\begin{bmatrix}
\varepsilon_{11,2} \\
\varepsilon_{22,2} \\
2\varepsilon_{12,2}
\end{bmatrix}
=B_{yy}u
\]
\newline
where matrices $ B_{xx} $ and $ B_{yy} $ contain linear combination of the second derivatives of the components of the shape function matrix with respect to x and y respectively. By using above expressions the higher order stress is obtain as
\[
\mu_{1ij}=
\begin{bmatrix}
\mu_{111} \\
\mu_{122} \\
\mu_{112} 
\end{bmatrix}
=l^2C\varepsilon_{,1} , \quad   
\mu_{2ij}=
\begin{bmatrix}
\mu_{211} \\
\mu_{222} \\
\mu_{212} 
\end{bmatrix}
=l^2C\varepsilon_{,2}
\]
\newline
Finally, substituting the above expression into the virtual work Eq.(5), yields the finite element equation $Ku = F$. Here, the element stiffness matrix K is given by,
\begin{equation}
K = K_l + l^2(K_{xx}+K_{yy}),
\end{equation}  
where the matrices $ K_l $ , $ K_{xx} $ and $ K_{yy} $ are expressed as,
\begin{equation}
\begin{aligned}
K_l &= \int_A B^TCBdA \\
K_{xx} &= \int_A B^T_{xx}CB_{xx}dA \\
K_{yy} &= \int_A B^T_{yy}CB_{yy}dA 
\end{aligned}
\end{equation}
here, as observed from Eq.(8), the general stiffness matrix of the strain gradient element ($C^1$ continuous element) consists of the two parts, which are basic ($K_l$) and a higher order one ($K_{xx}+K_{yy}$). From this it can be analyse that when the microstructural length parameter l is zero this Eq.(8) is reduced to the classical one.
\newline
\newline
This element has been implemented into the FE program ABAQUS using the User Element Subroutine UEL by Tomislav Lesicar and et al. They used the reduced Gauss integration
technique with 13 integration points for numerical integration of the stiffness matrices and force vector, instead of the full integration scheme with 25 points. The positions of the all 13 integration points are given in Fig. However, as discussed earlier this reduced integration technique for $C^1$ Continuous planar Triangular element provides not quite satisfactory results and it is more convenient for the multi scale analysis like Strain-gradient second-order computational homogenization scheme.     
\subsubsection{Physical interpretation of Strain Gradient}
Here, we can see the Physical interpretation of Strain Gradient \cite{lesivcar2017two}.
\begin{figure}[H]
	\begin{center}
		\includegraphics[scale=0.8]{U111_Eta_111.JPG}  \qquad \qquad
		\includegraphics[scale=0.8]{U222_Eta_222.JPG}
	\end{center}
   	\caption{Physical Interpretation of Strain Gradient $\eta_{111}$ and $\eta_{222}$}
\end{figure}
\begin{figure}[H]
	\begin{center}
		\includegraphics[scale=0.8]{U122_Eta_122.JPG}  \qquad \qquad
		\includegraphics[scale=0.8]{U211_Eta_211.JPG}
   	\caption{Physical Interpretation of Strain Gradient $\eta_{221}$ and $\eta_{112}$}		
	\end{center}  
\end{figure}
\begin{figure}[H]
	\begin{center}
		\includegraphics[scale=0.8]{U121_Eta_121.JPG}  \qquad \qquad
		\includegraphics[scale=0.8]{U212_Eta_212.JPG}
   	\caption{Physical Interpretation of Strain Gradient $\eta_{211}$ and $\eta_{122}$}
	\end{center}  
\end{figure}

\subsection{E. Amanatidou - Mixed type FEM formulation}
In Mindlin's 1960s theory (2.1), when the problem is formulated in terms of displacements, the governing partial differential Eq.(\ref{first}) is of fourth order. If traditional finite elements are used for the numerical solutions of such problems, then $C^1$ displacements continuity is required at inter elements.
\newline
\newline
Furthermore, E. Amanatidou \cite{amanatidou2002mixed} and et al. developed the alternative Mixed type finite element formulation, in which both displacements and the displacement gradients are used as independent unknowns and their relationship is enforced in an integral sense. In addition to that, this variational formulation can be used for both linear and non-linear strain gradient elasticity theories. In conclusion, this finite elements requires only $C^0$ continuity and simple to formulate and implement into FE program using UEL.   
\newline
\newline
The three equivalent forms strain energy density W given by Mindlin is expressed as,
\begin{equation}
W = \tilde{\mathbf{W}}(\varepsilon,\tilde{\mathbf{\kappa}}) = \hat{\mathbf{W}}(\varepsilon,\hat{\mathbf{\kappa}}) = \bar{\mathbf{W}}(\varepsilon,\bar{\kappa},\bar{\bar \kappa})
\end{equation}
\newline
where the expression $W = \tilde{\mathbf{W}}(\varepsilon,\tilde{\mathbf{\kappa}})$ known as "Type $I$", the expression $W =\hat{\mathbf{W}}(\varepsilon,\hat{\mathbf{\kappa}}) $ known as "Type $II$" and the expression $W = \bar{\mathbf{W}}(\varepsilon,\bar{\kappa},\bar{\bar \kappa}) $ known as "Type $III$".
\newline
\newline
where the Strain energy density for all three forms are represented as,
\newline
\newline
\begin{equation}\label{eleven}
\tilde{\mathbf{W}}(\varepsilon,\tilde{\mathbf{\kappa}}) = \dfrac{1}{2}\lambda\varepsilon_{ii}\varepsilon_{kk} + \mu\varepsilon_{ij}\varepsilon_{ij}+\frac{1}{2}l^2[\lambda\tilde{\kappa_{ijj}}\tilde{\kappa_{ikk}} + \mu(\tilde{\kappa_{ijk}}\tilde{\kappa_{ijk}}+\tilde{\kappa_{ijk}}\tilde{\kappa_{kji}})]
\end{equation}
\begin{equation}
 \hat{\mathbf{W}}(\varepsilon,\hat{\mathbf{\kappa}}) = \dfrac{1}{2}\lambda\varepsilon_{ii}\varepsilon_{kk} + \mu\varepsilon_{ij}\varepsilon_{ij}+\frac{1}{2}l^2(\lambda\tilde{\kappa_{ijj}}\tilde{\kappa_{ikk}}+2\mu\lambda\tilde{\kappa_{ijk}}\tilde{\kappa_{ijk}})
\end{equation}
\begin{equation}
\begin{aligned}
\bar{\mathbf{W}}(\varepsilon,\bar{\kappa},\bar{\bar \kappa}) &= \dfrac{1}{2}\lambda\varepsilon_{ii}\varepsilon_{kk} + \mu\varepsilon_{ij}\varepsilon_{ij} +l^2 [\frac{2}{9}(\lambda+3\mu)\bar{\kappa_{ij}}\bar{\kappa_{ij}}-\frac{2}{9} \lambda \bar{\kappa_{ij}} \bar{\kappa_{ji}} \\
& + \dfrac{1}{2}\lambda \bar{\bar{\kappa_{iij}}} \bar{\bar {\kappa_{kkj}}} + \mu \bar{\bar{ \kappa_{ijk}}} \bar{\bar{ \kappa_{ijk}}} \frac{2}{3}\lambda e_{ijk}\bar{\kappa_{ij}}\bar{\bar {\kappa_{kpp}}}]
\end{aligned}
\end{equation}
\newline
\newline
Amanatidou and Aravas only considered the first form of strain energy density Eq.(\ref{eleven}), which only depends upon conventional strain $ \varepsilon_{ij} $ and higher-order strain-gradient $ \kappa_{ijk} $. 
\newline
\subsubsection{ Stress  }
From the Eq.(\ref{eleven}) we can easily derived the Cauchy stress $\sigma_{ij}$ as,
\begin{equation}\label{fifteen}
\begin{aligned}
\bar{\sigma}_{ij} = \frac{\partial \tilde{\mathbf{W}} }{\partial \varepsilon_{ij}} &=  \lambda\varepsilon_{kk}\delta_{ij} +2\mu\varepsilon_{ij} 
\\
\\
&= \lambda\varepsilon_{kl}\delta_{kl}\delta_{ij} + 2\mu\varepsilon_{kl}\delta_{ik}\delta_{jl} 
\\
\\
&= (\lambda\delta_{kl}\delta_{ij} + 2\mu\delta_{ik}\delta_{jl})\varepsilon_{kl}
\end{aligned}
\end{equation} 
\newline
\newline
Now, it can be easily defined the Second-order tensor $C_{ijkl}$ using the stress-strain relation as,
\begin{equation}\label{sixteen}
{\sigma}_{ij} = C_{ijkl} \varepsilon_{kl} 
\end{equation}
By comparing the Eq.(\ref{fifteen}) and Eq.(\ref{sixteen}),
\begin{equation}\label{sixteen}
C_{ijkl} = \lambda\delta_{kl}\delta_{ij} + 2\mu\delta_{ik}\delta_{jl}
\end{equation}
It can be also written in symmetric $3\times3$ Matrix as,
\begin{equation}\label{seventeen}
C = 
\begin{bmatrix}
\lambda+2\mu & \lambda & 0 \\
\lambda & \lambda+2\mu & 0 \\
0 & 0 & \frac{1}{2}\mu
\end{bmatrix}
\end{equation}
\newline
\subsubsection { Higher Order Stress  }
From the Eq.(\ref{eleven}) we can easily derived the Higher Order stress $\mu_{ijk}$ as,
\begin{equation}\label{eighteen}
\begin{aligned}
\tilde{\mu}_{ijk} =  \frac{\partial \tilde{\mathbf{W}} }{\partial \kappa_{ijk}} &= \frac{1}{2}l^2[\lambda \kappa_{ijj}\delta_{ip} \delta_{kq} \delta_{kr} + \lambda \kappa_{ikk}\delta_{ip} \delta_{jq} \delta_{jr} \\
& + 2\mu \kappa_{ijk}\delta_{ip} \delta_{jq} \delta_{kr}  + \mu \kappa_{kji}\delta_{ip} \delta_{jq} \delta_{kr} + \mu \kappa_{ijk}\delta_{kp} \delta_{jq} \delta_{ir} ] \\
\\
&= \frac{1}{2}l^2 [ \lambda \kappa_{pjj}\delta_{qr} + \lambda \kappa_{pkk}\delta_{qr}  +2\mu \kappa_{pqr} + \mu\kappa_{rqq} + \mu\kappa_{rqp}   ] \\ 
\\
&= \frac{1}{2}l^2 [ \lambda\delta_{qr} (\kappa_{pjj}+\kappa_{pkk})   + 2\mu (\kappa_{pqr} + \kappa_{rqp}) ] \\
\\
&= \frac{1}{2}l^2 [ 2\lambda \delta_{qr} \delta_{ip} \delta_{kj} + 2\mu(\delta_{ip} \delta_{jq} \delta_{kr} + \delta_{ir} \delta_{jq} \delta_{kp})  ] \kappa_{ijk}
\\
\\
&= l^2 [ \lambda \delta_{qr} \delta_{ip} \delta_{kj} + \mu(\delta_{ip} \delta_{jq} \delta_{kr} + \delta_{ir} \delta_{jq} \delta_{kp}) ] \kappa_{ijk}
\end{aligned}
\end{equation}
\\
\\
Now, it can be easily derived the Third-order tensor $D_{pqrijk}$ using the Higher order stress and Strain gradient relation as,
\begin{equation}\label{nineteen}
\mu_{pqr} = D_{pqrijk} \kappa_{ijk}
\end{equation}
\\
\\
By, comparing the Eq.(\ref{eighteen}) and Eq.(\ref{nineteen}),
\begin{equation}\label{twenty}
D_{pqrijk} = l^2 [ \lambda \delta_{qr} \delta_{ip} \delta_{kj} + \mu(\delta_{ip} \delta_{jq} \delta_{kr} + \delta_{ir} \delta_{jq} \delta_{kp}) ]
\end{equation}
\\
\\
Now, Eq.(\ref{twenty}) can also be written in matrix notation as following, where $D_{pqrijk}$ is a symmetric $6\times6$ matrix.
\\
\\
\begin{equation}\label{twoone}
D = l^2
\begin{bmatrix}
\lambda + 2\mu & 0 & 0 & 0 & 0 & \frac{\lambda}{2} \\
0 & \mu & 0  & 0  & 0  & \frac{\mu}{2} \\
0 & 0 & \mu & 0  & \frac{\mu}{2}  & 0 \\
0 & 0 & 0 & \lambda + 2\mu & \frac{\lambda}{2} & 0 \\
0 & 0 & \frac{\mu}{2}  & \frac{\lambda}{2} & \frac{\lambda + 3\mu}{4} & 0 \\
 \frac{\lambda}{2} & \frac{\mu}{2} & 0 & 0 & 0 & \frac{\lambda + 3\mu}{4} 
\end{bmatrix}
\end{equation}
\\
\\
So, Eq.(\ref{seventeen}) and Eq.(\ref{twoone}) are the Voigt-notation of the constitutive equation of a general strain gradient solid introduced by Amanatidou and Aravas. 

\newpage
\subsubsection{ Mixed Tpye Elements}
    \begin{figure}[H]
    	\begin{center}
		     \includegraphics[scale=.65]{Element_mixed_type_formulation_E_Amanatidou.JPG}  	
	    \end{center}  
        \caption{C0-continuous Finite elements for Type I formulation.}   
    \end{figure}
Amanatidou and Aravas presented several elements that can be used in Type-I formulation are shown in above Fig. The elements are shown in Fig. , the corresponding nodal degrees of freedom are u,$\alpha$ and $\sigma$. Out of these given elements only the element I9-70 passes the all patch test, whereas the elements I5-28 and I13-70 failed into the patch test.
\\
\subsection{John Y. Shu - Mixed type FEM formulation}
\vspace{0.4cm}
Conventional continuum mechanics theories assume that stress at a material point is a function
of state variables, such as strain, at the same point. This assumption has valid until when the wavelength of a deformation field is much larger than the dominant micro-structural length scale of the material. However, when the two length scales are comparable, this assumption is proved wrong because the material behaviour at any material point is affected by the surrounding material points deformation. Therefore, Fleck—Hutchinson strain gradient plasticity, which falls within the Toupin—Mindlin framework, represent the virtual work in terms of strain gradients and higher order stresses.
\newline
\newline
John Y. Shu \cite{shu1999finite} developed Mixed type FEM formulation of Fleck—Hutchinson strain gradient elasticity theory. They devised $C^0$ continuous elements of mixed type, in which additional nodal degrees of freedom "Relaxed Strain" is introduced and enforce the kinematic constraints between displacement and relaxed strain by Lagrange multipliers.
\newline
\subsubsection{ Modified Virtual Work}
Firstly, for mixed type FEM formulation, they first derive a weak form of the principle of virtual
work suitable for finite element implementation using $C^0$-shape functions. In this modified virtual work only first-order gradients of kinematic quantities involving. In this introduced a second-order tensor $\psi$ and a related third-order tensor $\eta$ such that $\eta_{ijk}$ is defined as,
\\
\begin{equation}\label{twotwo} 
\eta_{ijk} = (\psi_{jk,i}+\psi_{ik,j})/2
\end{equation}
Now, modified weak form virtual work is represented as,
\begin{equation}\label{twothree}
\begin{aligned}
\int\limits_\Omega\! [ \sigma_{ij}\delta\varepsilon_{ij} + \tau_{ijk}\delta\eta_{ijk} + \tau_{ijk,i}(\delta\psi_{jk}-\delta u_{k,j})  ] d\Omega &= \int\limits_\Omega\! [b_k\delta u_k] d\Omega + \int\limits_\Gamma\! [ t_k\delta u_k + n_jr_k\delta\psi_{jk} ] d\Gamma \\
&+ \int\limits_\Gamma\! ( n_i\tau_{ijk} -n_jr_k ) ( \delta\psi_{jk}-\delta u_{k,j} ) d\Gamma
\end{aligned}
\end{equation}
\newline
for arbitrary variations of du and d$\psi$. If $ \psi $ is subjected strictly to the constraint of
$ \delta\psi = \delta u_{k,j} $ into the whole domain $ \Omega $, then as previously discussed, the strict enforcement of this constraint will demand $C^1$ continuous elements. Therefore, to facilitate the use of convenient $C^0$-continuous elements, this constraint is enforced in the following weighted residual manner as,
\begin{equation}\label{twofour}
\int\limits_\Omega\! (\psi_{jk} - u_{k,j})\delta\tau_{ijk,i} d\Omega = 0  \quad (no \quad sum\quad over \quad j \quad and \quad k)
\end{equation}
for an arbitrary variation of the Lagrange multipliers $ \tau_{ijk,i} $. Finally, by denoting the Lagrange multipliers $ \delta \tau_{ijk,i}$ in above equation as $ \delta\rho_{jk} $ the modified virtual work statement Eq.(\ref{twothree}) becomes,
\begin{equation}
\int\limits_\Omega\! [ \sigma_{ij}\delta\varepsilon_{ij} + \tau_{ijk}\delta\eta_{ijk} + \rho_{jk}(\delta\psi_{jk}-\delta u_{k,j})] d\Omega = \int\limits_\Omega\! [b_k\delta u_k] d\Omega + \int\limits_\Gamma\! [ t_k\delta u_k + n_jr_k\delta\psi_{jk} ] d\Gamma
\end{equation}
and the modified constraint from Eq.(\ref{twofour}) becomes,
\begin{equation}
\int\limits_\Omega\! (\psi_{jk} - u_{k,j})\delta\rho_{jk}  d\Omega = 0  \quad (no \quad sum\quad over \quad j \quad and \quad k)
\end{equation} 
By considering the above criteria we get the strain energy density in terms of the relaxed strain gradient from Eq.(\ref{two}) as,
\begin{equation}\label{twoseven}
W = \frac{1}{2}\lambda\varepsilon_{ii}\varepsilon_{jj}+\mu\varepsilon_{ij}\varepsilon_{ij} +\frac{1}{2}\mu l^2 ( \eta_{ijk}\eta_{ijk} - \eta_{ijk}\eta_{kji} )
\end{equation}
\\
\subsubsection{ Stress}
From the Eq.(\ref{twoseven}) we can easily derived the Cauchy stress $\sigma_{ij}$ as,
\begin{equation}\label{twoeight}
\begin{aligned}
\bar{\sigma}_{ij} = \frac{\partial \tilde{\mathbf{W}} }{\partial \varepsilon_{ij}} &=  \lambda\varepsilon_{kk}\delta_{ij} +2\mu\varepsilon_{ij} = (\lambda\delta_{kl}\delta_{ij} + 2\mu\delta_{ik}\delta_{jl})\varepsilon_{kl} = C_{ijkl} \varepsilon_{kl}
\end{aligned}
\end{equation} 
So, it yields the same result as given by Eq.(\ref{fifteen}) and Eq.(\ref{sixteen}). Therefore, we can also write the matrix notation of $ C_{ijkl} $ and it also same as Eq.(\ref{seventeen}).
\subsubsection{ Higher-order Stress}
From the Eq.(\ref{twoseven}) we can easily derived the Higher-order stress $\tau_{ijk}$ as,
\begin{equation}\label{twonine}
\begin{aligned}
\tau_{pqr} & = \frac{\partial \tilde{\mathbf{W}} }{\partial \eta_{pqr}} \\
& = \frac{1}{2}\mu l^2 [ 2(\eta_{ijk} \delta_{ip} \delta_{jq} \delta_{kr}) - \eta_{kji} \delta_{ip} \delta_{jq} \delta_{kr} - \eta_{ijk} \delta_{kp} \delta_{jq} \delta_{ir} ] \\
& =  \frac{1}{2}\mu l^2 [ 2(\eta_{pqr}) - \eta_{rqp} -\eta_{rqp} ] \\
& = \mu l^2 [ \eta_{pqr} - \eta_{rqp} ] \\
& = \mu l^2 [ \delta_{lp} \delta_{mq} \delta_{nr} - \delta_{lr} \delta_{mq} \delta_{np} ]  \eta_{lmn}
\end{aligned}
\end{equation}
Now, it can be easily derived the Third-order tensor $D_{pqrlmn}$ using the Higher order stress $ \tau_{pqr} $ and Strain gradient $ \eta_{lmn} $ relation as,
\begin{equation}\label{thirty}
\tau_{pqr} = D_{pqrlmn} \cdot \eta_{lmn}
\end{equation}
By, comparing the Eq.(\ref{twonine}) and Eq.(\ref{thirty}),
\begin{equation}\label{threeone}
D_{pqrlmn} = \mu l^2 [ \delta_{lp} \delta_{mq} \delta_{nr} - \delta_{lr} \delta_{mq} \delta_{np} ]
\end{equation}
Now, Eq.(\ref{twonine}) can also be written in matrix notation as following, where $D_{pqrlmn}$ is a symmetric $6\times6$ matrix.
\begin{equation}\label{threetwo}
D = \mu l^2
\begin{bmatrix}
0 & 0 & 0 & 0 & 0 & 0  \\
0 & 1 & 0  & 0  & 0  & -\frac{1}{2} \\
0 & 0 & 1 & 0  & -\frac{1}{2}  & 0 \\
0 & 0 & 0 & 0 & 0 & 0 \\
0 & 0 & -\frac{1}{2}  & 0 & \frac{1}{4} & 0 \\
0 & -\frac{1}{2} & 0 & 0 & 0 & \frac{1}{4} 
\end{bmatrix}
\end{equation}
So, Eq.(\ref{seventeen}) and Eq.(\ref{threetwo}) are the Voigt-notation of the constitutive equation of a Couple strain gradient solid represented by JOHN Y. SHU, WAYNE E. KING and NORMAN A. FLECK. 
\subsubsection{ Isoparametric elements for the strain gradient solid}
\begin{figure}[H]
	\begin{center}
		\includegraphics[scale=0.6]{Shu_1_2.JPG}  \qquad
		\includegraphics[scale=0.6]{Shu_3.JPG}
	\end{center}  
\end{figure}
\begin{figure}[H]
	\begin{center}
		\includegraphics[scale=0.6]{Shu_4.JPG}  \qquad
		\includegraphics[scale=0.6]{Shu_5_6.JPG}
	\end{center}  
\end{figure}
\begin{figure}[H]
	\begin{center}
		\includegraphics[scale=0.8]{Shu_info.JPG}
	\end{center}  
    \caption{C0-continuous 6 Finite elements for Mixed type of formulation.}
\end{figure}
A total of six types of triangular element and quadrilateral element have been developed by JOHN Y. SHU for strain gradient theory. In which Displacement gradients are introduced as extra nodal degrees of freedom in terms of relaxed strain. The kinematic constraints between the relaxed strain and true gradients of displacement are enforced via Lagrange multipliers. All six types of element passed a patch test. However, he recommended QU34L4 elements for practical applications, since it gives precise accuracy than other elements.
\newpage

\section{FEM Implementation}
In subsection "John Y. Shu - Mixed type FEM formulation", there are 6 different triangular element and quadrilateral element represented out of which element QU34L4 is recommended for practical applications. so here we will use the QU34L4 element in order to implement strain gradient theory.
\\
\subsection{QU34L4 - ELEMENT}
This QU34L4 is Shown in below Fig. ,which contains total 9 nodes and 38 degrees of freedom. In this element there are two displacements DOF $ u_1 $ and $ u_2  $ at each nodes of it. There are newly introduced relaxed strain DOF $ \psi_{11} $,$ \psi_{21} $,$ \psi_{12} $ and $ \psi_{22} $ at each corner nodes of it. In addition to that there are also four Lagrange multipliers $ \rho_{11} $,  $ \rho_{21} $, $ \rho_{12} $ and $ \rho_{22} $,  which are assumed to be constant throughout the element. So, in QU34L4 Q stands for nine-noded isoparametric quadrilateral element, where U stands for number of DOF related to the displacement which are total 34 in this element, where L stands for Lagrange multipliers which are total 4 in this element. 
\begin{figure}[H]
	\begin{center}
		\includegraphics[scale=0.7]{ele.png}
	\end{center} 
    \caption{C0-continuous QU34L4 Element.} 
\end{figure}
Displacements $ u_1 $,$ u_2 $ are interpolated using standard quadratic shape functions in terms of the area co-ordinates. Relaxed displacement gradients $ \psi_{11} $,$ \psi_{21} $,$ \psi_{12} $ and $ \psi_{22} $ are interpolated linearly. The different interpolation is motivated by the consideration that the true displacement gradients $ \varepsilon_{ij} $ will depend linearly
on the co-ordinates if all sides of the element are straight.
\subsection{Degrees of Freedom(DOF)}
In this section we derive the finite element shape function and its derivatives for the different DOF - displacements and relaxed strain. In this we also defined matrices for langrange multipliers and also established the matrices containing the shape functions and its derivatives.
\subsubsection{ Displacement DOF}
As discussed in above section, there are two displacements DOF $u_1$ and $u_2$ at each nodes. From the weak of modified virtual work we can derive the relation between global displacements and nodal-displacement as,
\\
\\
\begin{equation}\label{threethree}
u^{(e)}=
\begin{bmatrix}
u_1^{(e)}\\
u_2^{(e)}
\end{bmatrix}
=N^{[u]^{\textbf{T}}}\cdot \hat{u}^{(e)}
\end{equation}
\\
\\
where $\hat{u}_{(e)}$ is defined in terms of vector as,
\begin{equation}\label{threefour}
\hat u^{(e)}=
\begin{bmatrix}
u_1^{(1)}&u_2^{(1)}&u_1^{(2)}&u_2^{(2)}&\cdots &\cdots u_1^{(9)} u_2^{(9)}
\end{bmatrix}
\end{equation}
\\
In QU34L4 element, as discussed earlier $u_i$ are interpolated using standard quadratic shape functions in terms of the area co-ordinates. So here we derived the shape functions for nine-nodded element in coordinates $\xi$ and $\omega$.
\\
\\
\begin{equation}\label{threefive}
\begin{aligned}
N^{[U]^{(1)}} &= \dfrac{1}{4}\xi(\xi-1)\omega(\omega-1)  \\
N^{[U]^{(2)}} &= \dfrac{1}{4}\xi(\xi+1)\omega(\omega-1)  \\
N^{[U]^{(3)}} &= \dfrac{1}{4}\xi(\xi+1)\omega(\omega+1)  \\
N^{[U]^{(4)}} &= \dfrac{1}{4}\xi(\xi-1)\omega(\omega+1)  \\
N^{[U]^{(5)}} &= \dfrac{1}{2}(1-\xi^2)\omega(\omega-1)   \\
N^{[U]^{(6)}} &= \dfrac{1}{2}\xi(\xi+1)(1-\omega^2)  \\
N^{[U]^{(7)}} &= \dfrac{1}{2}(1-\xi^2)\omega(\omega+1)  \\
N^{[U]^{(8)}} &= \dfrac{1}{2}\xi(\xi-1)(1-\omega^2) \\
N^{[U]^{(9)}} &= (1-\xi^2)(1-\omega^2)  \\
\end{aligned}
\end{equation}
\\
\\
From equations Eq.(\ref{threethree}), Eq.(\ref{threefour}) and Eq.(\ref{threefive}) we can assemble the shape matrix as following,
\begin{equation}\label{threesix}
N^{[u]^{T}}=
\begin{bmatrix}
N^{[u]^{(1)}}&0&N^{[u]^{(2)}}&0&\cdots &\cdots& N^{[u]^{(9)}}&0 \\
0&N^{[u]^{(1)}}&0&N^{[u]^{(2)}}&\cdots &\cdots& 0&N^{[u]^{(9)}}
\end{bmatrix}
\end{equation}
\newpage
Now we have shape functions, so in next step we derive the local derivatives of the shape functions with respect to $\xi$ and $\omega$. 
\\
\\
The local derivatives of shape function with respect to are coordinates $\xi$ and $\omega$ is given as, 
\begin{equation}\label{threeseven}
N^{[U]^{(i)}}_{,\xi} = \dfrac{\partial N^{[U]^{(i)}} }{\partial \xi}
\end{equation}
\begin{equation}\label{threeeight}
N^{[U]^{(i)}}_{,\omega} = \dfrac{\partial N^{[U]^{(i)}} }{\partial \omega}
\end{equation}
\\
\\
From Eq.(\ref{threeseven}) we can get the local derivatives w.r.t  $\xi$ as,
\begin{equation}\label{threenine}
\begin{aligned}
N_{,\xi}^{[U]^{(1)}} &= -\dfrac{1}{2}\xi\omega-\dfrac{1}{4}\omega^2+\dfrac{1}{4}\omega+\dfrac{1}{2}\xi\omega^2  \\
N_{,\xi}^{[U]^{(2)}} &= -\dfrac{1}{2}\xi\omega+\dfrac{1}{4}\omega^2-\dfrac{1}{4}\omega+\dfrac{1}{2}\xi\omega^2  \\
N_{,\xi}^{[U]^{(3)}} &= \dfrac{1}{2}\xi\omega+\dfrac{1}{4}\omega^2+\dfrac{1}{4}\omega+\dfrac{1}{2}\xi\omega^2 \\
N_{,\xi}^{[U]^{(4)}} &= \dfrac{1}{2}\xi\omega-\dfrac{1}{4}\omega^2-\dfrac{1}{4}\omega+\dfrac{1}{2}\xi\omega^2 \\
N_{,\xi}^{[U]^{(5)}} &= \xi\omega-\xi\omega^2 \\
N_{,\xi}^{[U]^{(6)}} &= \xi-\xi\omega^2+\dfrac{1}{2}-\dfrac{1}{2}\omega^2 \\
N_{,\xi}^{[U]^{(7)}} &= -\xi\omega-\xi\omega^2 \\
N_{,\xi}^{[U]^{(8)}} &= \xi-\xi\omega^2-\dfrac{1}{2}+\dfrac{1}{2}\omega^2 \\
N_{,\xi}^{[U]^{(9)}} &= -2\xi+2\xi\omega^2 \\
\end{aligned}
\end{equation}
\\
From Eq.(\ref{threeeight}) we can get the local derivatives w.r.t  $\omega$ as,
\begin{equation}\label{fourty}
\begin{aligned}
N_{,\omega}^{[U]^{(1)}} &= -\dfrac{1}{2}\xi\omega-\dfrac{1}{4}\xi^2+\dfrac{1}{4}\xi+\dfrac{1}{2}\xi^2\omega \\
N_{,\omega}^{[U]^{(2)}} & = \dfrac{1}{2}\xi\omega-\dfrac{1}{4}\xi^2-\dfrac{1}{4}\xi+\dfrac{1}{2}\xi^2\omega \\
N_{,\omega}^{[U]^{(3)}} &= \dfrac{1}{2}\xi\omega+\dfrac{1}{4}\xi^2+\dfrac{1}{4}\xi+\dfrac{1}{2}\xi^2\omega \\
N_{,\omega}^{[U]^{(4)}} &= -\dfrac{1}{2}\xi\omega+\dfrac{1}{4}\xi^2-\dfrac{1}{4}\xi+\dfrac{1}{2}\xi^2\omega \\
N_{,\omega}^{[U]^{(5)}} &= \dfrac{1}{2}\xi^2-\dfrac{1}{2}-\xi^2\omega+\omega \\
N_{,\omega}^{[U]^{(6)}} &= -\xi^2\omega-\xi\omega \\
N_{,\omega}^{[U]^{(7)}} &=  -\dfrac{1}{2}\xi^2+\dfrac{1}{2}-\xi^2\omega+\omega \\
N_{,\omega}^{[U]^{(8)}} &= -\xi^2\omega+\xi\omega \\
N_{,\omega}^{[U]^{(9)}} &= -2\omega+2\xi^2\omega \\
\end{aligned}
\end{equation}
\\
Taking the equations above into consideration, the displacement gradient in the element is derived as following,
\begin{equation}\label{fourone}
\nabla u^{(e)}=
\begin{Bmatrix}
	u_{1,1}^{(e)} \\
	\\
	u_{1,2}^{(e)} \\
	\\
	u_{2,1}^{(e)} \\
	\\
	u_{2,2}^{(e)} 
\end{Bmatrix}
=M^{[u]^{\textbf{T}}}u^{(e)}
\end{equation}
\\
So here in Eq.(\ref{fourone}), the $u_{1,2}$ and $u_{2,1}$ are not same as in consideration of normal strain, where we considered $\varepsilon_{1,2}$ and $ \varepsilon_{2,1}$ are same.
\\
\\
From Eq.(\ref{threefour}) and Eq.(\ref{fourone}), we can defined the Displacements Gradient Matrix as following, 
\begin{equation}\label{fourtwo}
M^{[u]^\textbf{T}}=
\begin{bmatrix}
N_{,1}^{[u]^{(1)}}&0&N_{,1}^{[u]^{(2)}}&0&\cdots &\cdots& N_{,1}^{[u]^{(9)}}&0 \\
0&N_{,2}^{[u]^{(1)}}&0&N_{,2}^{[u]^{(2)}}&\cdots &\cdots& 0&N_{,2}^{[u]^{(9)}} \\
N_{,1}^{[u]^{(1)}}&0&N_{,1}^{[u]^{(2)}}&0&\cdots &\cdots& N_{,1}^{[u]^{(9)}}&0 \\
0&N_{,2}^{[u]^{(1)}}&0&N_{,2}^{[u]^{(2)}}&\cdots &\cdots& 0&N_{,2}^{[u]^{(9)}}
\end{bmatrix}
\end{equation}
\\
\\
Now, we can derived the true strain matrix as following,
\begin{equation}\label{fourthree}
\varepsilon^{(e)}=
\begin{Bmatrix}
\varepsilon_{11}^{(e)} \\
\\
\varepsilon_{22}^{(e)}  \\
\\
\gamma_{12}^{(e)}  \\ 
\end{Bmatrix}
\approx B^{[u]^{\textbf{T}}}u^{(e)}
\end{equation}
\\
From Eq.(\ref{threefour}) and Eq.(\ref{fourthree}), we can defined the differential Matrix $B^u$, which contains the global derivatives of shape functions as following, 
\begin{equation}\label{fourfour}
B^{[u]^\textbf{T}}=
\begin{bmatrix}
	N_{,1}^{[u]^{(1)}}&0&N_{,1}^{[u]^{(2)}}&0&\cdots &\cdots& N_{,1}^{[u]^{(9)}}&0 \\
	0&N_{,2}^{[u]^{(1)}}&0&N_{,2}^{[u]^{(2)}}&\cdots &\cdots& 0&N_{,2}^{[u]^{(9)}} \\
	N_{,2}^{[u]^{(1)}}&N_{,1}^{[u]^{(1)}}&N_{,2}^{[u]^{(2)}}&N_{,1}^{[u]^{(2)}}&\cdots &\cdots& N_{,2}^{[u]^{(9)}}&N_{,1}^{[u]^{(9)}}
\end{bmatrix}
\end{equation}
\\
\\
In both the Eq.(\ref{fourtwo}) and Eq.(\ref{fourfour}), it can be seen that there $N_{,1}$ and $N_{,2}$ which are represented as $N_{,1}^{[u]^{(i)}} = \dfrac{\partial N^{[u]^{(i)}}}{\partial x_1}$ and $N_{,2}^{[u]^{(i)}} = \dfrac{\partial N^{[u]^{(i)}}}{\partial x_2}$, which are global derivatives of shape functions related to displacements DOF. which can not be calculated directly since the shape functions are set up in the local coordinate system.
\subsubsection{ Relaxed Strain DOF}
As discussed in above section, there are four relaxed strain DOF $\psi_{11}$,$\psi_{21}$,$\psi_{12}$ and $\psi_{22}$,  at each corned nodes. Now, we can derive the relation between global relaxed strain and nodal-relaxed strain as,
\\
\begin{equation}\label{fourfive}
\psi^{(e)}=
\begin{Bmatrix}
\psi_{11}^{(e)} \\
\\
\psi_{21}^{(e)} \\
\\
\psi_{12}^{(e)} \\
\\
\psi_{22}^{(e)}
\end{Bmatrix}
=N^{[\psi]^{\textbf{T}}} \hat{\psi}^{(e)}
\end{equation}
\\
\\
where, nodal relaxed strain vector $\hat{\psi}^{(e)}$ is defined as following,
\begin{equation}\label{foursix}
\hat{\psi}^{(e)^{\textbf{T}}}=
\begin{bmatrix}
\psi_{(11)}^{(1)}&\psi_{(21)}^{(1)}&\psi_{(12)}^{(1)}&\psi_{(22)}^{(1)}&\cdots &\cdots& \psi_{(11)}^{(4)}&\psi_{(21)}^{(4)}&\psi_{(12)}^{(4)}&\psi_{(22)}^{(4)}
\end{bmatrix}
\end{equation}
\\
Now, defined the shape functions for the relaxed strain. As discussed earlier, in QU34L4 element $\psi_i$ are interpolated using standard bi-linear shape functions in terms of the area coordinates $\xi$ and $\omega$.
\\
\\
\begin{equation}\label{fourseven}
\begin{aligned}
N^{[\psi]^{(1)}} &= \dfrac{1}{4}(1-\xi)(1-\omega) \\
N^{[\psi]^{(2)}} &= \dfrac{1}{4}(1+\xi)(1-\omega) \\ 
N^{[\psi]^{(3)}} &= \dfrac{1}{4}(1+\xi)(1+\omega) \\
N^{[\psi]^{(4)}} &= \dfrac{1}{4}(1-\xi)(1+\omega) \\
\end{aligned}
\end{equation}
\\
\\
Now, considering the equations Eq.(\ref{fourfive}), Eq.(\ref{foursix}) and Eq.(\ref{fourseven}) we can assemble the shape matrix for the relaxed strain degree of freedom as following,
\\
\\
\begin{equation}\label{foureight}
N^{[\psi]^\textbf{T}}=
\begin{bmatrix}
N^{{\psi}^{(1)}}&0&0&0&\cdots&\cdots&N^{{\psi}^{(4)}}&0&0&0 \\
0&N^{{\psi}^{(1)}}&0&0&\cdots&\cdots&0&N^{{\psi}^{(4)}}&0&0 \\
0&0&N^{{\psi}^{(1)}}&0&\cdots&\cdots&0&0&N^{{\psi}^{(4)}}&0 \\
0&0&0&N^{{\psi}^{(1)}}&\cdots&\cdots&0&0&0&N^{{\psi}^{(4)}} 
\end{bmatrix}
\end{equation}
\\
\\
Now we have shape functions for relaxed strain DOF Eq.(\ref{fourseven}), so in next step we derive the local derivatives of the shape functions with respect to $\xi$ and $\omega$. 
\\
The local derivatives of shape function with respect to are coordinates $\xi$ and $\omega$ is given as, 
\begin{equation}\label{fournine}
N^{[\psi]^{(i)}}_{,\xi} = \dfrac{\partial N^{[\psi]^{(i)}} }{\partial \xi}
\end{equation}
\begin{equation}\label{fifty}
N^{[\psi]^{(i)}}_{,\omega} = \dfrac{\partial N^{[\psi]^{(i)}} }{\partial \omega}
\end{equation}
\\
From Eq.(\ref{fournine}) we can get the local derivatives w.r.t  $\xi$ as,
\begin{equation}
\begin{aligned}
N_{,\xi}^{[\psi]^{(1)}} &= \dfrac{1}{4}(-1+\omega) \\
N_{,\xi}^{[\psi]^{(2)}} &= \dfrac{1}{4}(1-\omega) \\
N_{,\xi}^{[\psi]^{(3)}} &= \dfrac{1}{4}(1+\omega) \\
N_{,\xi}^{[\psi]^{(4)}} &= \dfrac{1}{4}(-1-\omega) \\
\end{aligned}
\end{equation}
\\
From Eq.(\ref{fifty}) we can get the local derivatives w.r.t  $\omega$ as,
\begin{equation}
\begin{aligned}
N_{,\omega}^{[\psi]^{(1)}} &= \dfrac{1}{4}(-1+\xi) \\
N_{,\omega}^{[\psi]^{(2)}} &= \dfrac{1}{4}(-1-\xi) \\
N_{,\omega}^{[\psi]^{(3)}} &= \dfrac{1}{4}(1+\xi) \\
N_{,\omega}^{[\psi]^{(4)}} &= \dfrac{1}{4}(1-\xi) \\
\end{aligned}
\end{equation}
\\
Now, we defined the strain gradient using the differential matrix Eq.(\ref{fivefour}) and nodal relaxed strain Eq.(\ref{foursix})  
\begin{equation}\label{fivethree}
\eta^{(e)}=
\begin{Bmatrix}
	\eta_{111}^{(e)} \\
	\\
	\eta_{221}^{(e)} \\
	\\
	\eta_{112}^{(e)} \\
	\\
	\eta_{222}^{(e)} \\
	\\
	\phi_{1}^{(e)} \\
	\\
	\phi_{2}^{(e)}
\end{Bmatrix}
\thickapprox B^{[\psi]^{\textbf{T}}}\psi^{(e)}
\end{equation}
\\
Form, considering Eq.(\ref{foursix}) and Eq.(\ref{fivethree}) now we can defined the differential matrix $B^\psi$ as following,
\begin{equation}\label{fivefour}
B^{[\psi]^\textbf{T}}=
\begin{bmatrix}
N_{,1}^{{[\psi]}^{(1)}}&0&0&0&\cdots&\cdots&N_{,1}^{{[\psi]}^{(4)}}&0&0&0 \\
0&N_{,2}^{{[\psi]}^{(1)}}&0&0&\cdots&\cdots&0&N_{,2}^{{[\psi]}^{(4)}}&0&0 \\
0&0&N_{,1}^{{[\psi]}^{(1)}}&0&\cdots&\cdots&0&0&N_{,1}^{{[\psi]}^{(4)}}&0 \\
0&0&0&N_{,2}^{{[\psi]}^{(1)}}&\cdots&\cdots&0&0&0&N_{,2}^{{[\psi]}^{(4)}} \\
N_{,2}^{{[\psi]}^{(1)}}&N_{,1}^{{[\psi]}^{(1)}}&0&0&\cdots&\cdots&N_{,2}^{{[\psi]}^{(4)}}&N_{,1}^{{[\psi]}^{(4)}}&0&0 \\
0&0&N_{,2}^{{[\psi]}^{(1)}}&N_{,1}^{{[\psi]}^{(1)}}&\cdots&\cdots&0&0&N_{,2}^{{[\psi]}^{(4)}}&N_{,1}^{{[\psi]}^{(4)}} 
\end{bmatrix}
\end{equation}
\\
In this above Eq.(\ref{fivefour}), it can be seen that there $N_{,1}^\psi$ and $N_{,2}^\psi$ which are represented as $N_{,1}^{[\psi]^{(i)}} = \dfrac{\partial N^{[\psi]^{(i)}}}{\partial x_1}$ and $N_{,2}^{[\psi]^{(i)}} = \dfrac{\partial N^{[\psi]^{(i)}}}{\partial x_2}$, which are global derivatives of shape functions related to displacements DOF. which can not be calculated directly since the shape functions are set up in the local coordinate system $ \xi$ and $ \omega$.
\subsubsection{ Langrange Multiplier}
The Lagrange multipliers in the element $ \rho^{(e)}$ are defined as,
\begin{equation}
\rho^{(e)}=
\begin{Bmatrix}
\rho_{11}^{(e)} \\
\\
\rho_{21}^{(e)} \\
\\
\rho_{12}^{(e)} \\
\\
\rho_{22}^{(e)}
\end{Bmatrix}
=N^{[\rho]^{\textbf{T}}}\rho^{(e)}
\end{equation}
\\
As discussed earlier in above section, it is assumed that  Lagrange multipliers constant throughout the element they are only existent at the middle node and therefore, the shape functions matrix for it defined as,
\begin{equation}
N^{[\rho]}=
\begin{bmatrix}
1&0&0&0 \\
\\
0&1&0&0 \\
\\
0&0&1&0 \\
\\
0&0&0&1
\end{bmatrix}
\end{equation}
\newpage
\section{Verification of model without Gradient term}
After completely programmed the Strain Gradient theory for FEM model into Fortran programming language. I tried to run this code with ABAQUS. And Surprisingly, I got the different errors, out of which main errors are Convergence errors, NAN error and singularities error. In conclusion of that I have decided to debug my code from the scratch. Therefore, I have decided to check for the simpler case which is Mechanical Displacement degree of freedom. So first i wanted to check that my code is correctly running for the Displacement degree if freedom.
\\
\\
So, I have just commented all the terms which is related to gradient term i.e. which is related to the relaxed strain and Lagrange degree of freedom. After that I have implement my code to the plate-720 geometry which is explained in detailed in later sections.
\\
\\ 
Now, I have implement the my code which is correctly running after the removing of many hurdles in terms of programming error. Here, I representing the my code results for simpler loading conditions and compare it with the ABAQUS results.  
\\
\\
So, here I represent the displacement results of ABAQUS with simpler loading i.e. $ 1 \% $ of displacement in positive Y direction with my results without including the gradient terms and any other degrees of freedom.
\begin{figure}[H]
	\begin{center}
		\includegraphics[scale=0.4]{ABAQUS_result.png}  
	\end{center}  
    \caption{Displacement visualization of ABAQUS Model.}
\end{figure}
\begin{figure}[H]
	\begin{center}
		\includegraphics[scale=0.4]{MY_code_result_1.png} 
	\end{center}  
    \caption{Displacement visualization of Strain Gradient Model.}
\end{figure}
So, here I represent the Stress results of ABAQUS with simpler loading i.e. $ 1 \% $ of displacement in positive Y direction with my results without including the gradient terms and any other degrees of freedom. In the second fig I have represent the complex part of geometry.
\begin{figure}[H]
	\begin{center}
		\includegraphics[scale=0.4]{ABAQUS_result_stress.png} 
		\includegraphics[scale=0.35]{ABAQUS_result_stress_crp.png} 
	\end{center}  
    \caption{Stress visualization of ABAQUS Model.}
\end{figure}
\begin{figure}[H]
	\begin{center}
		\includegraphics[scale=0.4]{MY_code_result_stress.png} \quad
		\includegraphics[scale=0.32]{MY_code_result_stress_crp.png}
	\end{center}  
   \caption{Stress visualization of Strain Gradient Model.}
\end{figure}

\newpage
\section{Verification of Model Plate-720}
As of described in E. Amanatidou - Mixed type FEM formulation, they have used the PLATE-720 which contains the 720 elements and 2983 nodes. Using this geometry they try to find the stress concentration factor and plot the stress concentration factor for different microstructural length.
\\
So, here I have used the same geometry i.e. Plate with domain length of 10 and hole radius is 0.333. I also mesh it as same as described in that paper with also transition of fine mesh into the coarse mesh, which is at 1.332 from the centre of the hole.   
\begin{figure}[H]
	\begin{center}
		\includegraphics[scale=0.35]{mesh_part.png} 
	\end{center}  
   \caption{Meshed Plate-720 with hole in the left-bottom corner.}
\end{figure}
Here, I specifically represent the fine mesh around the hole and also the transition of mesh from coarse mesh to fine mesh at specific position from the centre of the hole.
\begin{figure}[H]
	\begin{center}
		\includegraphics[scale=0.38]{fine_mesh_part.png} 
	\end{center} 
   \caption{Fine mesh and Transition in Plate-720.} 
\end{figure}
\begin{figure}[H]
	\begin{center}
		\includegraphics[scale=0.4]{Gradient_720.png} 
	\end{center}  
   \caption{Strain Gradient Visualization in Plate-720.}
\end{figure}
\begin{figure}[H]
	\begin{center}
		\includegraphics[scale=0.4]{Gradient_720_11.png} 
	\end{center}  
   \caption{Visualization of Strain Gradient $\eta_{111} $}
\end{figure}
\begin{figure}[H]
	\begin{center}
		\includegraphics[scale=0.4]{Gradient_720_12.png} 
	\end{center}  
   \caption{Visualization of Strain Gradient $\eta_{221} $}
\end{figure}
\begin{figure}[H]
	\begin{center}
		\includegraphics[scale=0.4]{Gradient_720_13.png} 
	\end{center}  
   \caption{Visualization of Strain Gradient $\eta_{112} $}
\end{figure}
\begin{figure}[H]
	\begin{center}
		\includegraphics[scale=0.4]{Gradient_720_22.png} 
	\end{center}  
   \caption{Visualization of Strain Gradient $\eta_{222} $}
\end{figure}
\begin{figure}[H]
	\begin{center}
		\includegraphics[scale=0.4]{Gradient_720_23.png} 
	\end{center}  
   \caption{Visualization of Strain Gradient $\eta_{121}$ + $\eta_{211}$}
\end{figure}
\begin{figure}[H]
	\begin{center}
		\includegraphics[scale=0.4]{Gradient_720_33.png} 
	\end{center}  
   \caption{Visualization of Strain Gradient $\eta_{122}$ + $\eta_{212}$}
\end{figure}

\newpage
\section{Patch Tests}
In this chapter I present the Patch test which is implemented with the one element and four element of same geometry using finite element method. The Patch tests are necessary for checking the stability of the newly developed user element and also important to check the convergence of the finite element solutions.  
\\
\subsection{Patch Test - 1}
Now, we considered an one Square geometry with specific dimensions as given below. In below given fig L = 1.0 and so therefore the coordinates of the points A(0,0), B(1,0), C(1,1) and D(0,1).  
\begin{figure}[H]
	\begin{center}
		\includegraphics[scale=0.6]{patch_one.JPG} 
	\end{center}  
   \caption{Square Domain with one element for implementation of Patch Test-1.}
\end{figure} 
In this Patch test we just find the analytical solution for displacements and relaxed strain.
After that we are finding the values of the displacements and relaxed strain at each nodes and each corner nodes respectively by assigning the position of that node. In the last we implement the all nodal degrees of freedom in terms of the boundary condition and we find the stiffness matrix and right hand side matrix for one element by running the code with ABAQUS and we try to prove the equilibrium equation.
\\
\\
In details, we have total 18 displacements values of all nodes and 16 relaxed strain values of each corner nodes of the element which are prescribe in the Input file for one element in terms of boundary condition.  
\\ 
\\
For proving the equilibrium equation, we have stiffness matrix, RHS matrix and displacement matrix in hand. But the dimension of the above matrices are big so we wouldn't prefer to solve it on paper by hand. so i write one python scripts which reads the data of all matrices from individual files. So for that now we have to put the stiffness matrix into 'AMATRX.dat', RHS matrix into 'RHS.dat' and displacement matrix into 'U.dat' file respectively. This python file read the data and store it in terms of multidimensional array and that try to find the residual of the equilibrium equation and print it into terminal so we can verify our results.
\begin{figure}[H]
	\begin{center}
		\includegraphics[scale=0.45]{p1_u1.png} 
	\end{center}  
    \caption{Visualization of Displacement $u_1 $ in Patch Test-1 Model.}
\end{figure}
\begin{figure}[H]
	\begin{center}
		\includegraphics[scale=0.45]{p1_u2.png} 
	\end{center}  
    \caption{Visualization of Displacement $u_2 $ in Patch Test-1 Model.}
\end{figure}
\subsection{Patch Test - 2}
Now, we considered the same geometry with same dimensions as described in Patch test-1, but now we mesh it into four element as shown in below fig.
\begin{figure}[H]
	\begin{center}
		\includegraphics[scale=0.75]{patch_four.JPG} 
	\end{center}  
    \caption{Square Domain with four element for implementation of Patch Test-2.}
\end{figure}
Now,we trying to implement the Patch test-2, for this one there are some criteria to pass the count test therefore, we considered the four element on the same geometry instead of one. In this patch test we just find the displacements and relaxed strain of the nodes at the boundary by analytical way as same as in patch test one. 
\\
\\
In details, we have total 32 displacements values of all nodes and 32 relaxed strains values at each corner nodes of the elements which are prescribed in the Input file for four element in terms of boundary conditions.
\\
\\
As same as in Patch Test-1 i wrote the one python script for patch test-2 which also gives the analytical values of displacements and relaxed strains of the deformed nodes. we just give the values of it at the boundary and we compare the values of all inner nodes with analytical solutions.
\begin{figure}[H]
	\begin{center}
		\includegraphics[scale=0.45]{p2_u1.png} 
	\end{center}
    \caption{Visualization of Displacement $u_1 $ in Patch Test-2 Model.}  
\end{figure}
\begin{figure}[H]
	\begin{center}
		\includegraphics[scale=0.45]{p2_u2.png} 
	\end{center}
    \caption{Visualization of Displacement $u_2 $ in Patch Test-2 Model.} 
\end{figure}
\newpage
\begin{appendices}
\section{Python file for Patch test-1}
\includegraphics[scale=0.93,page=1]{Patch_test_1_elem_eqn.pdf}
\includegraphics[scale=0.93,page=1]{pro.pdf}
\section{Python file for Patch test-2}
\includegraphics[scale=0.80,page=1]{Patch_Test_4_elem_eqn.pdf}
\includegraphics[scale=0.80,page=2]{Patch_Test_4_elem_eqn.pdf}
\section{Python file for 9-node Element mesh generation}
\includegraphics[scale=0.80,page=1]{9Node_Mesh_generation.pdf}
\includegraphics[scale=0.80,page=2]{9Node_Mesh_generation.pdf}
\section{Flowchart of Strain Gradient Elasticity FEM Model}
\vspace{5cm}
\includegraphics[scale=0.60,page=1]{flow_chart_1.pdf}
\includegraphics[scale=0.60,page=1]{flow_chart_2.pdf}
\end{appendices}





















\newpage
\bibliographystyle{plain}
\bibliography{refren}
\end{document}