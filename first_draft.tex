\documentclass[12pt]{article}

\usepackage{amsmath}
\usepackage{amsfonts}
\usepackage{amssymb}
\usepackage{graphicx}
\usepackage{caption}
\usepackage{mathtools}
\usepackage{lipsum}
\usepackage{stackengine}
\usepackage{fancyhdr}
\usepackage{caption}
\usepackage{tikz}
\usetikzlibrary{shapes.geometric, arrows}
\usepackage{float}
\usepackage[a4paper,left=1in,right=1in,top=1in,bottom=1in,footskip=.25in]{geometry}
\usepackage{etoolbox}
\usepackage[nottoc]{tocbibind}
\usepackage{tabu}
\usepackage{enumitem,kantlipsum}
\begin{document}
	\begin{titlepage}
		\centering
		
		\begin{figure}
			\begin{center}
				\includegraphics[scale=.2]{tubaf.pdf}  
			\end{center}
			
		\end{figure}
		
		
		
		%\includegraphics[width=0.15\textwidth]{download}\par\vspace{1cm}
		{\scshape\LARGE \textbf{Technische Universit\"at Bergakademie Freiberg} \par}
		\vspace{1cm}
		{\scshape\Large PERSONAL PROGRAMMING PROJECT\par}
		{\scshape\Large (PPP)\par}
		\vspace{1.5cm}
		{\huge\bfseries Implementation of Gradient Elasticity Model In FEM \par}
		\vspace{2cm}
		{\scshape\Large \textbf{Dhaval Rasheshkumar Patel}\par}
		{\scshape\Large 63940\par}
		\vfill
		{\normalsize\ Supervised by\par}
		
		Dr.~ \textsc{Sergii Kozinov}
		
		\vfill
		
		% Bottom of the page
		{\large \today\par}
	\end{titlepage}
	\clearpage
    \textbf{\LARGE Abstract}\\ \newline
    \newline


    \newpage
    \clearpage
    \tableofcontents
    \clearpage

\section{Introduction}
Classical continuum solid mechanics theories, such as linear or non-linear elasticity and plasticity, have been used in wide range of fundamental problems and applications in various fields, but "Classical Continuum Constitutive Models" possess no material/intrinsic scale. So in this regime of micron and nano-scales that experimental evidence and observations have suggested that classical continuum theories do not suffice for an accurate and detailed description in the modelling of size dependent phenomena. Moreover, classical elastic singularities as those emerging during the application of point loads and description of size effects also not solve by it.
\newline
\par
In General, in Gradient elasticity theories, the length scales enter the constitutive equations through the elastic strain energy function, which, in this case depends not only on the strain tensor but also on gradients of the strain tensors Gradient Elasticity theories provide extensions of the classical equations of elasticity with additional higher order spatial derivatives of strains and stresses, but in Gradient elasticity theory one of the most challenging task is to keep the number of additional constitutive parameters to a minimum.
\newline
\par
Stress equation of equilibrium, constitutive equations and boundary conditions of the "Strain Gradient theory" were first given in a non-linear form by Toupin in 1960s. After this famous Strain Gradient elasticity theory in three different forms proposed by Mindlin in 1960s. In such theories , when the problem is formulated in terms of displacements, the governing partial differential equation contains the fourth order derivative of displacements. If traditional finite element formulation are used for the numerical solution of such problems, then $C^{1}$ displacement continuity is required. $C^{1}$ displacement continuity means displacement and its first derivative is continuous in inter-element. However, in FEM there is no any robust $C^{1}$ continuous element. So an alternative mixed finite elements formulation is developed, in which both the displacement and the displacement gradients are used as independent unknowns and their relationship is enforced using Lagrange Multiplier Method. In 1999 JOHN Y. SHU proposed the mixed finite formulation based Toupin—Mindlin theories, in which only $C^{0}$ continuous element was used. After this in 2002 E. Amanatidou and N. Aravas also proposed the Mixed type FEM formulation for Toupin—Mindlin theories and also used $C^{0}$ continuous element.
\newline  
\par
In the present "Personal Programming Project" report, in the very first section we discussed the theoretical background of the strain gradient elasticity theory with its evolutions and different versions. Following to this section, we discussed in brief few of them theories. After this the Mixed type of FEM formulation with $C^{0}$ continuous element for Strain Gradient theory is discussed in details.
\newpage
 
\section{Theoretical Background}
In this Section the main focus is to understand the "Classical Strain Gradient Elasticity Theory" and motivation behind the evolution of Strain Gradient Elasticity theories. In this section the brief overview of Toupin's, Mindlin's, Eringen's and Aifantis theory is given. After this also discussed about different possible FEM formulation.  
     \begin{figure}[H]
     	\includegraphics[scale=0.6]{file.png}  
     \end{figure}
\subsection{Mindlin's 1964 Theory.}
In the early 1960s, the Mindlin presented a theory of elasticity with Microstructure length, in which he distinguished between kinematic quantities on two scale micro and macro. As discussed in introduction that it is very challenging task to keep the number of constitutive parameters to a minimum. In this theory constitutive tensor contains 1764 coefficients in total, but for isotropic material its reduced drastically to amount 18. Mindlin present the Strain energy equation as,
\begin{equation}\label{first}
\begin{aligned}
U= &\frac{1}{2}\lambda\varepsilon_{ii}\varepsilon_{jj}+\mu\varepsilon_{ij}\varepsilon_{ij}+\frac{1}{2}b_1 \gamma_{ii} \gamma_{jj}+\frac{1}{2}b_2 \gamma_{ij} \gamma_{ij}+\frac{1}{2}b_3 \gamma_{ij} \gamma_{ji}+g_1\gamma_{ii}\varepsilon_{jj}
\\   
&+g_2(\gamma_{ij}+\gamma_{ji})\varepsilon_{jj}+a_1\kappa_{iik}\kappa_{kjj}+a_2\kappa_{iik}\kappa_{jkj}+\frac{1}{2}a_3\kappa_{iik}\kappa_{jjk}
\\
&+\frac{1}{2}a_4\kappa_{iij}\kappa_{ikk}
 +a_5\kappa_{iij}\kappa_{kik}
+\frac{1}{2}a_8\kappa_{iji}\kappa_{kjk}+\frac{1}{2}a_{10}\kappa_{ijk}\kappa_{ijk}
\\
&+a_{11}\kappa_{ijk}\kappa_{jki}+\frac{1}{2}a_{13}\kappa_{ijk}\kappa_{ikj}
+\frac{1}{2}a_{14}\kappa_{ijk}\kappa_{jik}
+\frac{1}{2}a_{15}\kappa_{ijk}\kappa_{kji}
\end{aligned}
\end{equation}  
\newline
where $\lambda$ and $\mu$ are the usual Lame constants and the various $a_i$ , $b_i$ and $g_i$ are 16 additional constitutive coefficients. However for practical purpose the use of eq(1) is very limited as it requires so much additional coefficients. In later 1960s Mindlin also formulated the simpler version of his own elasticity theory by making assumption of expressing the strain energy density in terms of displacement only. So in last he proposed three different form of it. However, the equation of strain energy density is of order four and it requires $C^{1}$ continuity.
\newpage

\subsection{Aifanti's 1992 Theory.} 
In the early 1990s, motivated by own work in plasticity and non-linear elasticity, Aifantis suggested to extend the linear constitutive model given by Mindlin in form $II$, in which second order terms are expressed as the strain gradient tensor. So he modified this Mindlin form $II$ by neglecting the most of the gradient coefficients. Mindlin present that for a general isotropic elastic solid the strain energy density depends upon
$\varepsilon_{ij}$ and $\kappa_{ijk}$ as the following,
\begin{equation}
\begin{aligned}
W(\varepsilon,\kappa) = 
& G(\varepsilon_{ij}\varepsilon_{ij}+\frac{\nu}{1-2\nu}\varepsilon_{ij}\varepsilon_{ij}) \\
&    +a_1\kappa_{iik}\kappa_{kjj}+a_2\kappa_{iik}\kappa_{kjj}+a_3\kappa_{iik}\kappa_{kjj}+a_4\kappa_{iik}\kappa_{kjj}+a_5\kappa_{iik}\kappa_{kjj}
\end{aligned}
\end{equation}
where G is elastic shear modulus, $\nu$ is Poisson's ratio and $a_1$,$a_2$,$a_3$,$a_4$,$a_5$ are material constants.   
\newline
In this he considered,
\newline
\begin{equation}
\begin{aligned}
& a_1 = a_3 = a_5 = 0,   &   a_2 = \frac{\nu}{1-2\nu}Gl^2, &&  a_4 = Gl^2
\end{aligned}
\end{equation} 
Accordingly, the strain energy density is defined as
\begin{equation}
W = \frac{1}{2}\lambda\varepsilon_{ii}\varepsilon_{jj}+\mu\varepsilon_{ij}\varepsilon_{ij}+l^2(\frac{1}{2}\lambda\varepsilon_{ii,k}\varepsilon_{jj,k}+\mu\varepsilon_{ij,k}\varepsilon_{ij,k})
\end{equation}
In Eq. (4), $\lambda$ and $\mu$ are Lame constants, while $l$ represents microstructural parameter (material length scale).However this equation(4) of strain energy also requires 
$C^{1}$ continuous element.
\newline
\newline
\textbf{Conclusion of above both theories (2.1) and (2.2) : }
\newline
\newline
In above both strain gradient theories the principle of virtual work for a linear elastic strain gradient solid can be expressed as
\begin{equation}
\int\limits_\upsilon\! [\sigma_{ij} \delta\varepsilon_{ij} + \tau_{ijk}\delta\eta_{ijk}]  dV = \int\limits_\upsilon\! [b_k\delta u_k] dV + \int\limits_s \! [f_k\delta u_k+r_kD\delta u_k]dS
\end{equation}
where $D(.) = n_k\frac{\partial(.)}{\partial x_k}$ is surface normal-gradient operator, $b_k$ is the body force per unit volume
of the body $V$ while $f_k$ and $r_k$ are the truncation and the double stress traction per unit area of the surface $S$. They are in equilibrium with the Cauchy stress $\sigma_{ij}$ and the higher-order stress $\tau_{ijk}$ according to
\begin{equation}
b_k + (\sigma_{ij}-\tau_{jik,j})_{,i} = 0
\end{equation}
The constitutive law governing the stress $\sigma_{ij}$ and the higher-order stress $\tau_{ijk}$ for an elastic solid is derived through $\sigma_{ij} = \partial w / \partial \varepsilon_{ij}$ and $\tau_{ijk}= \partial w / \partial \eta_{ijk}$ where w is the strain energy density per unit volume. In this Cauchy stress $\sigma_{ij}$ is work conjugate to the strain $\varepsilon_{ij}$ and higher-order stress $\tau_{ijk}$ is work conjugate to the strain gradient  $\eta_{ijk}$.
\newline
\newline
Here, second-order derivatives of displacement occurred in the principle of virtual work Eq.(5), implying that displacement-based elements of $C^1$-continuity are indispensable in a finite element formulation. However there are no robust $C^1$ continuous elements were available at that time for the application of fem formulation of above mentioned both strain gradient theories.

\section{Different FEM Approach}
There are different FEM approach to the Strain Gradient Elasticity theories are available because of the fact that $C^1$ continuous elements are very difficult to formulate and on the other hand there are also a mixed finite element formulation of strain gradient elasticity derived, which only requires $C^0$ continuity.

\subsection{Tomislav Lesicar - Two scale FEM formulation} 
\subsubsection{ C1-Continuous Element}
The Aifantis strain gradient theory given in subsection (2.2) has been embedded into finite element framework by Tomislav Lesicar, Zdenko Tonkovic and Jurica Soric. They used the three node triangular finite element named C1PE3. The element is shown in
Fig. It contains twelve degrees of freedom (DOF) per node, and it satisfies $C^1$ continuity with assumptions of the plane strains with unit thickness
\newline
     \begin{figure}[H]
     	\begin{center}
	      \includegraphics[scale=0.8]{Tri_Element.JPG}  
	    \end{center}  
     \end{figure}
\subsubsection{ Higher-order stress and strain gradient tensor} 
They used the same weak form of the finite element formulation as given by Eq.(5). Furthermore, using the basic finite element relations strain and stress tensors can be expressed as
\begin{equation}
\varepsilon = Bu,      \sigma=C\varepsilon
\end{equation}
\newline
where B is a matrix containing linear combinations of the first derivatives of the components of the shape function matrix, ̄C is an isotropic elastic constitutive matrix and u  is a displacement vector. Relating to Eq.(7) the strain gradient tensor is represented as

\[
\varepsilon_{,1}=
\begin{bmatrix}
\varepsilon_{11,1} \\
\varepsilon_{22,1} \\
2\varepsilon_{12,1} 
\end{bmatrix}
=B_{xx}u , \quad   
\varepsilon_{,2}=
\begin{bmatrix}
\varepsilon_{11,2} \\
\varepsilon_{22,2} \\
2\varepsilon_{12,2}
\end{bmatrix}
=B_{yy}u
\]
\newline
where matrices $ B_{xx} $ and $ B_{yy} $ contain linear combination of the second derivatives of the components of the shape function matrix with respect to x and y respectively. By using above expressions the higher order stress is obtain as
\[
\mu_{1ij}=
\begin{bmatrix}
\mu_{111} \\
\mu_{122} \\
\mu_{112} 
\end{bmatrix}
=l^2C\varepsilon_{,1} , \quad   
\mu_{2ij}=
\begin{bmatrix}
\mu_{211} \\
\mu_{222} \\
\mu_{212} 
\end{bmatrix}
=l^2C\varepsilon_{,2}
\]
\newline
Finally, substituting the above expression into the virtual work Eq.(5), yields the finite element equation $Ku = F$. Here, the element stiffness matrix K is given by,
\begin{equation}
K = K_l + l^2(K_{xx}+K_{yy}),
\end{equation}  
where the matrices $ K_l $ , $ K_{xx} $ and $ K_{yy} $ are expressed as,
\begin{equation}
\begin{aligned}
K_l &= \int_A B^TCBdA \\
K_{xx} &= \int_A B^T_{xx}CB_{xx}dA \\
K_{yy} &= \int_A B^T_{yy}CB_{yy}dA 
\end{aligned}
\end{equation}
here, as observed from Eq.(8), the general stiffness matrix of the strain gradient element ($C^1$ continuous element) consists of the two parts, which are basic ($K_l$) and a higher order one ($K_{xx}+K_{yy}$). From this it can be analyse that when the microstructural length parameter l is zero this Eq.(8) is reduced to the classical one.
\newline
\newline
This element has been implemented into the FE program ABAQUS using the User Element Subroutine UEL by Tomislav Lesicar and et al. They used the reduced Gauss integration
technique with 13 integration points for numerical integration of the stiffness matrices and force vector, instead of the full integration scheme with 25 points. The positions of the all 13 integration points are given in Fig. However, as discussed earlier this reduced integration technique for $C^1$ Continuous planar Triangular element provides not quite satisfactory results and it is more convenient for the multi scale analysis like Strain-gradient second-order computational homogenization scheme.     

\subsection{E. Amanatidou - Mixed type FEM formulation}
In Mindlin's 1960s theory (2.1), when the problem is formulated in terms of displacements, the governing partial differential Eq.(\ref{first}) is of fourth order. If traditional finite elements are used for the numerical solutions of such problems, then $C^1$ displacements continuity is required at inter elements.
\newline
\newline
Furthermore, E. Amanatidou and et al. developed the alternative Mixed type finite element formulation, in which both displacements and the displacement gradients are used as independent unknowns and their relationship is enforced in an integral sense. In addition to that, this variational formulation can be used for both linear and non-linear strain gradient elasticity theories. In conclusion, this finite elements requires only $C^0$ continuity and simple to formulate and implement into FE program using UEL.   
\newline
\newline
The three equivalent forms strain energy density W given by Mindlin is expressed as,
\begin{equation}
W = \tilde{\mathbf{W}}(\varepsilon,\tilde{\mathbf{\kappa}}) = \hat{\mathbf{W}}(\varepsilon,\hat{\mathbf{\kappa}}) = \bar{\mathbf{W}}(\varepsilon,\bar{\kappa},\bar{\bar \kappa})
\end{equation}
where the expression $W = \tilde{\mathbf{W}}(\varepsilon,\tilde{\mathbf{\kappa}})$ known as "Type $I$", the expression $W =\hat{\mathbf{W}}(\varepsilon,\hat{\mathbf{\kappa}}) $ known as "Type $II$" and the expression $W = \bar{\mathbf{W}}(\varepsilon,\bar{\kappa},\bar{\bar \kappa}) $ known as "Type $III$".
\newline
\newline
where the Strain energy density for all three forms are represented as,
\begin{equation}\label{eleven}
\tilde{\mathbf{W}}(\varepsilon,\tilde{\mathbf{\kappa}}) = \dfrac{1}{2}\lambda\varepsilon_{ii}\varepsilon_{kk} + \mu\varepsilon_{ij}\varepsilon_{ij}+\frac{1}{2}l^2[\lambda\tilde{\kappa_{ijj}}\tilde{\kappa_{ikk}} + \mu(\tilde{\kappa_{ijk}}\tilde{\kappa_{ijk}}+\tilde{\kappa_{ijk}}\tilde{\kappa_{kji}})]
\end{equation}
\begin{equation}
 \hat{\mathbf{W}}(\varepsilon,\hat{\mathbf{\kappa}}) = \dfrac{1}{2}\lambda\varepsilon_{ii}\varepsilon_{kk} + \mu\varepsilon_{ij}\varepsilon_{ij}+\frac{1}{2}l^2(\lambda\tilde{\kappa_{ijj}}\tilde{\kappa_{ikk}}+2\mu\lambda\tilde{\kappa_{ijk}}\tilde{\kappa_{ijk}})
\end{equation}
\begin{equation}
\begin{aligned}
\bar{\mathbf{W}}(\varepsilon,\bar{\kappa},\bar{\bar \kappa}) &= \dfrac{1}{2}\lambda\varepsilon_{ii}\varepsilon_{kk} + \mu\varepsilon_{ij}\varepsilon_{ij} +l^2 [\frac{2}{9}(\lambda+3\mu)\bar{\kappa_{ij}}\bar{\kappa_{ij}}-\frac{2}{9} \lambda \bar{\kappa_{ij}} \bar{\kappa_{ji}} \\
& + \dfrac{1}{2}\lambda \bar{\bar{\kappa_{iij}}} \bar{\bar {\kappa_{kkj}}} + \mu \bar{\bar{ \kappa_{ijk}}} \bar{\bar{ \kappa_{ijk}}} \frac{2}{3}\lambda e_{ijk}\bar{\kappa_{ij}}\bar{\bar {\kappa_{kpp}}}]
\end{aligned}
\end{equation}
\newline
\newline
Amanatidou and Aravas only considered the first form of strain energy density Eq.(\ref{eleven}), which only depends upon conventional strain $ \varepsilon_{ij} $ and higher-order strain-gradient $ \kappa_{ijk} $. 
\newline
\newline
\subsubsection{ Stress  }
From the Eq.(\ref{eleven}) we can easily derived the Cauchy stress $\sigma_{ij}$ as,
\begin{equation}\label{fifteen}
\begin{aligned}
\bar{\sigma}_{ij} = \frac{\partial \tilde{\mathbf{W}} }{\partial \varepsilon_{ij}} &=  \lambda\varepsilon_{kk}\delta_{ij} +2\mu\varepsilon_{ij} 
\\
\\
&= \lambda\varepsilon_{kl}\delta_{kl}\delta_{ij} + 2\mu\varepsilon_{kl}\delta_{ik}\delta_{jl} 
\\
\\
&= (\lambda\delta_{kl}\delta_{ij} + 2\mu\delta_{ik}\delta_{jl})\varepsilon_{kl}
\end{aligned}
\end{equation} 
\newline
\newline
Now, it can be easily defined the Second-order tensor $C_{ijkl}$ using the stress-strain relation as,
\begin{equation}\label{sixteen}
{\sigma}_{ij} = C_{ijkl} \varepsilon_{kl} 
\end{equation}
By comparing the Eq.(\ref{fifteen}) and Eq.(\ref{sixteen}),
\begin{equation}
C_{ijkl} = \lambda\delta_{kl}\delta_{ij} + 2\mu\delta_{ik}\delta_{jl}
\end{equation}
It can be also written in symmetric $3\times3$ Matrix as,
\begin{equation}\label{seventeen}
C = 
\begin{bmatrix}
\lambda+2\mu & \lambda & 0 \\
\lambda & \lambda+2\mu & 0 \\
0 & 0 & \frac{1}{2}\mu
\end{bmatrix}
\end{equation}
\newpage
\subsubsection { Higher Order Stress  }
From the Eq.(\ref{eleven}) we can easily derived the Higher Order stress $\mu_{ijk}$ as,
\begin{equation}\label{eighteen}
\begin{aligned}
\tilde{\mu}_{ijk} =  \frac{\partial \tilde{\mathbf{W}} }{\partial \kappa_{ijk}} &= \frac{1}{2}l^2[\lambda \kappa_{ijj}\delta_{ip} \delta_{kq} \delta_{kr} + \lambda \kappa_{ikk}\delta_{ip} \delta_{jq} \delta_{jr} \\
& + 2\mu \kappa_{ijk}\delta_{ip} \delta_{jq} \delta_{kr}  + \mu \kappa_{kji}\delta_{ip} \delta_{jq} \delta_{kr} + \mu \kappa_{ijk}\delta_{kp} \delta_{jq} \delta_{ir} ] \\
\\
&= \frac{1}{2}l^2 [ \lambda \kappa_{pjj}\delta_{qr} + \lambda \kappa_{pkk}\delta_{qr}  +2\mu \kappa_{pqr} + \mu\kappa_{rqq} + \mu\kappa_{rqp}   ] \\ 
\\
&= \frac{1}{2}l^2 [ \lambda\delta_{qr} (\kappa_{pjj}+\kappa_{pkk})   + 2\mu (\kappa_{pqr} + \kappa_{rqp}) ] \\
\\
&= \frac{1}{2}l^2 [ 2\lambda \delta_{qr} \delta_{ip} \delta_{kj} + 2\mu(\delta_{ip} \delta_{jq} \delta_{kr} + \delta_{ir} \delta_{jq} \delta_{kp})  ] \kappa_{ijk}
\\
\\
&= l^2 [ \lambda \delta_{qr} \delta_{ip} \delta_{kj} + \mu(\delta_{ip} \delta_{jq} \delta_{kr} + \delta_{ir} \delta_{jq} \delta_{kp}) ] \kappa_{ijk}
\end{aligned}
\end{equation}
\\
\\
Now, it can be easily derived the Third-order tensor $D_{pqrijk}$ using the Higher order stress and Strain gradient relation as,
\begin{equation}\label{nineteen}
\mu_{pqr} = D_{pqrijk} \kappa_{ijk}
\end{equation}
\\
\\
By, comparing the Eq.(\ref{eighteen}) and Eq.(\ref{nineteen}),
\begin{equation}\label{twenty}
D_{pqrijk} = l^2 [ \lambda \delta_{qr} \delta_{ip} \delta_{kj} + \mu(\delta_{ip} \delta_{jq} \delta_{kr} + \delta_{ir} \delta_{jq} \delta_{kp}) ]
\end{equation}
\\
\\
Now, Eq.(\ref{twenty}) can also be written in matrix notation as following, where $D_{pqrijk}$ is a symmetric $6\times6$ matrix.
\\
\\
\begin{equation}\label{twoone}
D = l^2
\begin{bmatrix}
\lambda + 2\mu & 0 & 0 & 0 & 0 & \frac{\lambda}{2} \\
0 & \mu & 0  & 0  & 0  & \frac{\mu}{2} \\
0 & 0 & \mu & 0  & \frac{\mu}{2}  & 0 \\
0 & 0 & 0 & \lambda + 2\mu & \frac{\lambda}{2} & 0 \\
0 & 0 & \frac{\mu}{2}  & \frac{\lambda}{2} & \frac{\lambda + 3\mu}{4} & 0 \\
 \frac{\lambda}{2} & \frac{\mu}{2} & 0 & 0 & 0 & \frac{\lambda + 3\mu}{4} 
\end{bmatrix}
\end{equation}
\\
\\
So, Eq.(\ref{seventeen}) and Eq.(\ref{twoone}) are the Voigt-notation of the constitutive equation of a general strain gradient solid introduced by Amanatidou and Aravas. 

\newpage
\subsubsection{ Mixed Tpye Elements}
    \begin{figure}[H]
    	\begin{center}
		     \includegraphics[scale=.65]{Element_mixed_type_formulation_E_Amanatidou.JPG}  	
	    \end{center}     
    \end{figure}
Amanatidou and Aravas presented several elements that can be used in Type-I formulation are shown in above Fig. The elements are shown in Fig. , the corresponding nodal degrees of freedom are u,$\alpha$ and $\sigma$. Out of these given elements only the element I9-70 passes the all patch test, whereas the elements I5-28 and I13-70 failed into the patch test.
\\
\\
\subsection{John Y. Shu - Mixed type FEM formulation}
\vspace{0.4cm}
Conventional continuum mechanics theories assume that stress at a material point is a function
of state variables, such as strain, at the same point. This assumption has valid until when the wavelength of a deformation field is much larger than the dominant micro-structural length scale of the material. However, when the two length scales are comparable, this assumption is proved wrong because the material behaviour at any material point is affected by the surrounding material points deformation. Therefore, Fleck—Hutchinson strain gradient plasticity, which falls within the Toupin—Mindlin framework, represent the virtual work in terms of strain gradients and higher order stresses.
\newline
\newline
John Y. Shu developed Mixed type FEM formulation of Fleck—Hutchinson strain gradient elasticity theory. They devised $C^0$ continuous elements of mixed type, in which additional nodal degrees of freedom "Relaxed Strain" is introduced and enforce the kinematic constraints between displacement and relaxed strain by Lagrange multipliers.
\newline
\newline
Firstly, for mixed type FEM formulation, they first derive a weak form of the principle of virtual
work suitable for finite element implementation using $C^0$-shape functions. In this modified virtual work only first-order gradients of kinematic quantities involving. In this introduced a second-order tensor $\psi$ and a related third-order tensor $\eta$ such that $\eta_{ijk}$ is defined as,
\\
\begin{equation}\label{twotwo} 
\eta_{ijk} = (\psi_{jk,i}+\psi_{ik,j})/2
\end{equation}
Now, modified weak form virtual work is represented as,
\begin{equation}\label{twothree}
\begin{aligned}
\int\limits_\Omega\! [ \sigma_{ij}\delta\varepsilon_{ij} + \tau_{ijk}\delta\eta_{ijk} + \tau_{ijk,i}(\delta\psi_{jk}-\delta u_{k,j})  ] d\Omega &= \int\limits_\Omega\! [b_k\delta u_k] d\Omega + \int\limits_\Gamma\! [ t_k\delta u_k + n_jr_k\delta\psi_{jk} ] d\Gamma \\
&+ \int\limits_\Gamma\! ( n_i\tau_{ijk} -n_jr_k ) ( \delta\psi_{jk}-\delta u_{k,j} ) d\Gamma
\end{aligned}
\end{equation}
\newline
for arbitrary variations of du and d$\psi$. If $ \psi $ is subjected strictly to the constraint of
$ \delta\psi = \delta u_{k,j} $ into the whole domain $ \Omega $, then as previously discussed, the strict enforcement of this constraint will demand $C^1$ continuous elements. Therefore, to facilitate the use of convenient $C^0$-continuous elements, this constraint is enforced in the following weighted residual manner as,
\begin{equation}\label{twofour}
\int\limits_\Omega\! (\psi_{jk} - u_{k,j})\delta\tau_{ijk,i} d\Omega = 0  \quad (no \quad sum\quad over \quad j \quad and \quad k)
\end{equation}
for an arbitrary variation of the Lagrange multipliers $ \tau_{ijk,i} $. Finally, by denoting the Lagrange multipliers $ \delta \tau_{ijk,i}$ in above equation as $ \delta\rho_{jk} $ the modified virtual work statement Eq.(\ref{twothree}) becomes,
\begin{equation}
\int\limits_\Omega\! [ \sigma_{ij}\delta\varepsilon_{ij} + \tau_{ijk}\delta\eta_{ijk} + \rho_{jk}(\delta\psi_{jk}-\delta u_{k,j})] d\Omega = \int\limits_\Omega\! [b_k\delta u_k] d\Omega + \int\limits_\Gamma\! [ t_k\delta u_k + n_jr_k\delta\psi_{jk} ] d\Gamma
\end{equation}
and the modified constraint from Eq.(\ref{twofour}) becomes,
\begin{equation}
\int\limits_\Omega\! (\psi_{jk} - u_{k,j})\delta\rho_{jk}  d\Omega = 0  \quad (no \quad sum\quad over \quad j \quad and \quad k)
\end{equation} 













\end{document}